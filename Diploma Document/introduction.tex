\chapter{Introduction}\label{ch:intro}
%these sections are optional, up-to the author
We know that Kazakh people are well-endowed from their own nature. They have huge potential to acquire new skills in short stretch of time, they can pretend for the bright future if they keep up on progressing and developing their national values. However, we shouldn't leave off just with national values. In order to be able to offer something valuable to the regional and global markets nation and every person comprising the nation should be competitive. Competitive not in terms of only material goods, but also in terms of knowledge, intellectual products and quality of human resources. 

Thus, we aimed to create a platform which will either contribute to the significant area of nation's development and help to uplift national spirit in the education sector. The project which will make learners look over gaining knowledge from the other angle, through other lens. The platform that will be in use of every school children's academic and daily life.

We came up with an idea to create an Educational Learning Management Platform, which enables students to track their own progress and to prioritize tasks. I can say this is our main foremost attribute. Why did we focus on it so much? Because, our progress, our accomplishments - are what give us energy. Energy to conquest, energy to achieve more accomplishments, energy to do significant realizations. This is an idea that comes from well-known Kaizen planning system. According to this system we should see our all life-through performances, acts, deeds in a visual desk.\cite{singh2009kaizen} Thus, we built on function into our product, which is called progress tracking.

Moreover, according to the article written by Richelle V. Adams1, Erik Blair published in journal "Sage" \ref{ch:Ref}: successful time management is associated with greater academic performance and lower levels of anxiety in students. Consequently, we decided to add some features for time management. 

It is difficult to overestimate the opportunities that online education systems offer to students of our time. Equal access to information, low barriers to knowledge acquisition, mobility and constantly updated material are the main advantages of the platforms. And nowadays there are a lot of online courses and platforms for use in universities and schools. 

Online educational platforms are able to form independence and cooperation skills in the process of building online network interactions, which is beneficial for a modern person, since these skills are valued much more in the process of modern communication than the discipline and competition developed by classical educational models. 

The idea of our platform is that it provides the school community with 
1. the platform that will make school learners remember that they're citizens of Kazakhstan, by adding national patterns in national style.
2. the platform enriched with features that didn't exist in educational school platforms before, where each task/project is supported by extension in the form of multi-level tasks, the completion of which contribute to consistent and systematic learning of skills. 

It is interesting to note the fact that 93\% of studies found at least one engagement dimension \& 50\% at least one disengagement dimension in the learning approach was found according to the research held in the University College London \cite{bond2020facilitating}. And also highlighting the claim that teacher-created videos are more likely to lead to task completion & overall engagement is also worth it \cite{bond2020facilitating}. So, from this we get an  assumption that there should be an update or transformation in learning approach at schools. Because, school is the place where children make the first impression about gaining education.  

We hope to contribute to youth development by facilitating their learning process, making it more creative and engaging; and accustoming them to plan their activities, goals, to keep them motivated by providing them with motivational citates while they're using our product. In conclusion, we want all Kazakh children to achieve their goals, to set clear expectations for themselves, always to be full of beans. Overall, we expect them to contribute to the development of our motherland and to be an honest, well-educated citizen. 

\section{Problem statement}
In our country we don't have a platform for school staff, with new functionalities, new design,  For instance, at the moment there are platforms for universities, where students can see the whole content/topics that will be taught during the study of the subject in advance. But school learners lacking this opportunity. This is just one drop of a problem that we found out during our research. Analysis showed us that there were some cases even which represent a reputational risk.\\ 
Moreover, there is a new trend, according to which cartoons, films, accessories, clothes, comics are produced in a way they contribute to our national values, by creating their own modern kazakh style. Furthermore, the circumstances of our national values are hanging in the balance nowadays, so there should be a place for maintaining Kazakh style in educational platforms as well.\\
Before creating our project we defined specific range of questions that arose in our minds. \\
\textbf{What additional functions do teachers and children need during the work with online platforms? }\\
\textbf{What functions are hard to be found in the online education environment?}\\
\textbf{How to improve the level of productivity of school children?}\\
We need to build a platform that will bridge the gaps in online learning environment at schools of Kazakhstan. 
\newpage
\section{Aims and Objectives}
\subsubsection{Aim:}
Organize learning system in a new functional way by maintaining national style .
\subsubsection{Objectives: }
Investigations on existing platforms will be performed.\\
Time tracker will be created.\\
Secure plagiarism checker will be developed. \\
User friendly webpage in user friendly, slightly national design will be built. \\
Downsides of existing platforms will be added to our's as strengths.
\section{Thesis Outline}
Theme: "Subject management platform: Easy Mektep". \\
The main goal of the diploma project was to solve some problems of virtual education system of Kazakhstan. To tackle this, we brainstormed, made research and created a platform with essential, in demand functions which were acknowledged by questionnaire results.\\
The diploma project includes an introduction, problem statement, requirements, planning, methodologies, final conclusion, future iterations, list of references and literature.\\
Keywords: distance learning, online classes, MySQL, PHP, Anti-plagiarism, PYTHON, sklearn, cosine-similarity.
